\documentclass[12pt]{article}
\usepackage{hyperref}
\hypersetup{
    colorlinks=true,
    linkcolor=blue,
    filecolor=magenta,      
    urlcolor=cyan,
    pdftitle={Overleaf Example},
    pdfpagemode=FullScreen,
    }
\usepackage{xcolor}
\usepackage{amsmath,amssymb,latexsym,epsfig,amsthm}
\usepackage{graphicx}
\usepackage{url}
\usepackage{listings}
\addtolength{\hoffset}{-2.5 cm} \addtolength{\textwidth}{5 cm}
\topmargin-0in \textheight9.2in
\parindent0pt
\usepackage{setspace}
\linespread{1}
\newtheorem{theorem}{Theorem}[section]
\newtheorem{definition}[theorem]{Definition}
\newtheorem{proposition}[theorem]{Proposition}
\newtheorem{example}[theorem]{Example}
\newtheorem{lemma}[theorem]{Lemma}
\newtheorem{corollary}[theorem]{Corollary}
\newtheorem{claim}[theorem]{Claim}
\newtheorem{rem}[theorem]{Remark}
\newtheorem{algorithm}[theorem]{Algorithm}
%\newmathop{Hom}
\newcommand{\Hom}{\operatorname{Hom}}
\newcommand{\End}{\operatorname{End}}
\newcommand{\BLambda}{\boldsymbol{\Lambda}}
\newcommand{\BSigma}{\boldsymbol{\Sigma}}
\newcommand{\beps}{\boldsymbol{\eps}}
\newcommand{\bbeta}{\boldsymbol{\beta}}
\newcommand{\btheta}{\boldsymbol{\theta}}
\newcommand{\BA}{\boldsymbol{A}}
\newcommand{\BJ}{\boldsymbol{J}}
\newcommand{\BH}{\boldsymbol{H}}
\newcommand{\BU}{\boldsymbol{U}}
\newcommand{\BV}{\boldsymbol{V}}
\newcommand{\BS}{\boldsymbol{S}}
\newcommand{\eps}{\epsilon}
\newcommand{\BX}{\boldsymbol{X}}
\newcommand{\BF}{\boldsymbol{F}}
\newcommand{\BY}{\boldsymbol{Y}}
\newcommand{\bp}{\boldsymbol{p}}
\newcommand{\bq}{\boldsymbol{q}}
\newcommand{\bw}{\boldsymbol{w}}
\newcommand{\bu}{\boldsymbol{u}}
\newcommand{\bv}{\boldsymbol{v}}
\newcommand{\bx}{\boldsymbol{x}}
\newcommand{\by}{\boldsymbol{y}}
\newcommand{\bz}{\boldsymbol{z}}
\newcommand{\I}{\boldsymbol{I}}
\newcommand{\bone}{\boldsymbol{1}}
\newcommand{\RR}{\mathbb{R}}
\newcommand{\pa}{\partial}

%\newcommand{\ker}{\operatorname{Ker}}
\newcommand{\im}{\operatorname{Im}}
\DeclareMathOperator{\Var}{Var}
\DeclareMathOperator{\MSE}{MSE}
\DeclareMathOperator{\bias}{bias}
\DeclareMathOperator{\Tr}{Tr}
\DeclareMathOperator{\se}{se}
\DeclareMathOperator{\Cov}{Cov}
\DeclareMathOperator{\rank}{rank}
\DeclareMathOperator{\Proj}{Proj}
\DeclareMathOperator{\diag}{diag}
\DeclareMathOperator{\argmax}{argmax}
\DeclareMathOperator{\E}{\mathbb{E}}
\newcommand{\Ans}{\\{\bf Answer: }}
\renewcommand{\Pr}{\mathbb{P}}
\newcommand*\diff{\mathop{}\!\mathrm{d}}

\begin{document}
     
{\bf \centerline{\Large Midterm Review}}
\vskip0.5cm

\begin{enumerate}
\item Brownian Motions
\\
Property 1: 
\[
dW\sim\mathcal N(0,dt)
\]
\begin{enumerate}
\item Multiplication rules (3.10.1 [Shreve])
\[
dW(t)dW(t)=dt, dW(t)dt=0, dtdt=0
\]
\item Derive the formula for $d(W^2)$, $d(W^4)$.
\end{enumerate}
Property 2: all the increment $dW$s are independent from each other.
\begin{enumerate}
\item Suppose $X(T)=\int_0^T f(x)dt + \int_0^T g(x)dW_t$. Find $Var(X)$
\end{enumerate}
\newpage
\item Ito's Formula
\\
\begin{enumerate}
\item 1D case
\[
df(t,x) = f_tdt+f_xdx + \frac12 f_{xx}dxdx
\]
 Compute the stochastic differential $dZ$ when 
\begin{enumerate}
\item $Z(t) = exp(\alpha t)$
\item $Z(t) = exp(\alpha X(t))$ with 
\[
dX(t) = \mu dt + \sigma dB(t)
\]
\item
$Z(t)=1/X(t)$ with 
\[
dX(t) = aX(t)dt + \sigma X(t) dW(t)
\]
\end{enumerate}
\item 2D case
\[
df(t,x,y) = f_tdt+(f_xdx+f_ydy) + \frac12( f_{xx}dxdx+2f_{xy}dxdy+f_{yy}dydy )
\]

Derive the Ito's product rule $d(XY)=XdY+YdX+dXdY$

\end{enumerate}
\newpage
\item Geometric Brownian Motions
\[dS(t) = \alpha(t)S(t)dt+\sigma(t)S(t)dW(t), 0\leq t\leq T\]
Set 
\[ D(t) = exp\left(-\int_0^t R(s)ds\right)\]
\begin{enumerate}
\item Derive a formula for $S(t)$
\item Derive a formula for $d(D(t)S(t))$ by Ito's product rule.
\item Derive a formula for $d(D(t)S(t))$ by Ito's formula (Exercise 5.1 [Shreve]). 
\\
Hint: Consider $f(x) = S(0)e^x$ and set 
\[
X(t) = \int_0^t \sigma(s)dW(s) + \int_0^t \left(\alpha(s)-R(s)-\frac12\sigma^2(s) \right)
\]
\item Show that $S$ is log-normally distributed. i.e., show that $log(S)$ is normally distributed.
\end{enumerate}
\newpage
\item Black-Scholes-Merton Equation
\\
Let $c(t,x)$ denote the value of an option at time $t$ with current price $S(t)=x$. A profolio $X(t)$ with hedging strategy $\Delta(t)$ should satisfy
\[
d(e^{-rt}X(t)) = d(e^{-rt}c(t,x))
\]
Use Ito's formula to compute both sides to get
\[
\left\{
\begin{array}{l}
\Delta(t) = c_x \\
rc = c_t +rxc_x + \frac12 \sigma^2 x^2 c_{xx}
\end{array}
\right.
\]
For European call options, we have
\[
c(t,0) = 0
\]
\[
c(T,x) = max(S(T)-K, 0)
\]
We solve the equation with the boundary conditions above to get 
\[
c(t,x)=xN(d+)-Ke^{-rt}N(d-),
\]
where
\[
d_{\pm} = \frac1{\sigma\sqrt{T-t}} \left(log \frac xK + \left(r\pm \frac {\sigma^2}{2}(T-t) \right) \right)
\]
The problems about the BSM equations could be extremely difficult. For a reference, you may take a look at the problem as an excerpt of Exercise 4.9 [Shreve]
\begin{enumerate}
\item Show that for $x>K$, $\lim_{t\to T^-}d_{\pm}=\infty$, but for $0< x< K$,  $\lim_{t\to T^-}d_{\pm}=-\infty$
\item Show that for $0\leq t<T$,  $\lim_{t\to 0^+}d_{\pm}=-\infty$
\item Show that for $0\leq t<T$,  $\lim_{t\to \infty}d_{\pm}=\infty$.
\item Use (c) to verify \[ lim_{x\to\infty} \left(c(t,x)-(x-e^{-r(T-t)K})\right)=0 \]. 
\\
Hint: In this verification, you will need to show that 
\[
\lim_{x\to\inf} \frac{N(d_+)-1}{x^{-1}}.
\]
Use the L'Hopital's rule and the fact
\[
x=K exp\left(\sigma\sqrt{T-t}d_+ - (T-t)\left(r+\frac12\sigma^2\right) \right)
\]
\end{enumerate}
\end{enumerate}

\end{document}




















