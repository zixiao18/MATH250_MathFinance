\documentclass{article}
\pagestyle{empty}
\usepackage{amsmath,amssymb,latexsym,epsfig,amsthm}
\usepackage{graphicx}
\usepackage{url}
\addtolength{\hoffset}{-2.5 cm} \addtolength{\textwidth}{5 cm}
\topmargin-0in \textheight9.2in
\parindent0pt
\newtheorem{theorem}{Theorem}[section]
\newtheorem{definition}[theorem]{Definition}
\newtheorem{proposition}[theorem]{Proposition}
\newtheorem{example}[theorem]{Example}
\newtheorem{lemma}[theorem]{Lemma}
\newtheorem{corollary}[theorem]{Corollary}
\newtheorem{claim}[theorem]{Claim}
\newtheorem{rem}[theorem]{Remark}
\newtheorem{algorithm}[theorem]{Algorithm}
%\newmathop{Hom}
\newcommand{\Hom}{\operatorname{Hom}}
\newcommand{\End}{\operatorname{End}}

\newcommand{\BLambda}{\boldsymbol{\Lambda}}
\newcommand{\BSigma}{\boldsymbol{\Sigma}}
\newcommand{\beps}{\boldsymbol{\eps}}
\newcommand{\bbeta}{\boldsymbol{\beta}}
\newcommand{\btheta}{\boldsymbol{\theta}}
\newcommand{\BA}{\boldsymbol{A}}
\newcommand{\BJ}{\boldsymbol{J}}
\newcommand{\BH}{\boldsymbol{H}}
\newcommand{\BU}{\boldsymbol{U}}
\newcommand{\BV}{\boldsymbol{V}}
\newcommand{\BS}{\boldsymbol{S}}
\newcommand{\eps}{\epsilon}
\newcommand{\BX}{\boldsymbol{X}}
\newcommand{\BY}{\boldsymbol{Y}}
\newcommand{\bp}{\boldsymbol{p}}
\newcommand{\bq}{\boldsymbol{q}}
\newcommand{\bu}{\boldsymbol{u}}
\newcommand{\bv}{\boldsymbol{v}}
\newcommand{\bx}{\boldsymbol{x}}
\newcommand{\by}{\boldsymbol{y}}
\newcommand{\bz}{\boldsymbol{z}}
\newcommand{\I}{\boldsymbol{I}}
\newcommand{\bone}{\boldsymbol{1}}
\newcommand{\RR}{\mathbb{R}}
\newcommand{\pa}{\partial}

%\newcommand{\ker}{\operatorname{Ker}}
\newcommand{\im}{\operatorname{Im}}
\DeclareMathOperator{\Var}{Var}
\DeclareMathOperator{\MSE}{MSE}
\DeclareMathOperator{\bias}{bias}
\DeclareMathOperator{\Tr}{Tr}
\DeclareMathOperator{\se}{se}
\DeclareMathOperator{\Cov}{Cov}
\DeclareMathOperator{\rank}{rank}
\DeclareMathOperator{\Proj}{Proj}
\DeclareMathOperator{\diag}{diag}
\DeclareMathOperator{\argmax}{argmax}
\DeclareMathOperator{\E}{\mathbb{E}}
\newcommand{\Ans}{\\{\bf Answer: }}
\renewcommand{\Pr}{\mathbb{P}}
\newcommand*\diff{\mathop{}\!\mathrm{d}}

\begin{document}

{\bf \centerline{\Large Mathematics of Finance }}
\vskip0.25cm
{\bf \centerline{\Large Exercises 1 \quad Brownian Motions}}
\vskip0.25cm

\begin{itemize}
\item {\bf Change in Office Hours - 3.30 - 5pm on Mondays at CIWW 805 (in person only), or by appointment. The office hours on Wednesdays are cancelled.}
\item {\bf Recommended Textbook - Stochastic Calculus for Finance, Vol II, by S.E. Shreve; Springer Verlag. }
\item {\bf Problems on the handout are drawn from or inspired by the exercises in the book.} 
\end{itemize}

\begin{enumerate}
\item (Exercise 4.7 [Shreve] Calculations on Brownian Motions)
\begin{enumerate}
\item Compute $dW^4$ and then write $W^4$ as the sum of an ordinary (Lebesgue) integral.
\item Take expectations on both sides to derive the formula $\mathbb EW^4(T)=3T^2$.
\item Deduce a formula for $\mathbb EW^6$.
\end{enumerate}
\item (Exercise 4.19 [Shreve]) Let $W(t)$ be a Brownian motion and define
\[
B(t)=\int_0^t sign(W(s))dW(s),
\]
where 
\[
sign(x)=\begin{cases} 1 & x\geq0, \\ -1 & x<0 \end{cases}
\]
\begin{enumerate}
\item Show that $(dB(t))^2=dt$. Hence $B(t)$ is a Brownian motion by Levy's theorem.
\item Show the \emph{It$\hat o$'s product rule} $d(XY)=XdY+YdX+dXdY$ for stochastic process $X(t), Y(t)$. 
\item Use (b) to compute $d(B(t)W(t))$. Conclude that $B(t)$ and $W(t)$ are uncorrelated normal random variables by showing $\mathbb E(B(t)W(t))=0$. 
\item Compute $dW^2(t)$ and conclude that $B(t)$ and $W(t)$ are not independent by showing $\mathbb E[B(t)W^2(t)] \neq \mathbb EB(t)\cdot \mathbb EW^2(t)$. Why does this happen to uncorrelated normal variables?
\end{enumerate}
\item (Geometric Brownian Motions)
Assume a stock price be a geometric Brownian motion \[dS(t) = \alpha S(t)dt + \sigma S(t)dW(t)\] 
\begin{enumerate}
\item Apply the It$\hat o$'s lemma to solve for $S$.
\item Compute $d(S^p(t))$.
\end{enumerate}
(Exercise 4.18 [Shreve]) Let $X$ denote the value of an investor's profolio with a hedging strategy of $\Delta(t)$.
\begin{enumerate}
\addtocounter{enumii}{2}
\item Find $dX$.
\end{enumerate}
Denote $\theta = (\alpha-r)/\sigma$ as the \emph{market price of risk}, where \emph{r} denotes the interest rate. Define the \emph{state price desity process} as 
$\zeta(t) = exp\left\{ -\theta W(t) - \left(r+ \theta^2/2\right)t \right\}$. 
\begin{enumerate}
\addtocounter{enumii}{3}
\item Find $d\zeta$. Hint: use two different ways to express
$d(e^{rt}\zeta)$
\item Show that $\zeta(t)X(t)$ is a martingale. (i.e. $d(\zeta(t)X(t))$ has no $dt$-terms). 
\end{enumerate}
From (c), the \emph{present value} at $t=0$ of the random payment $V(T)$ at $t=T$ is $X(0)=\mathbb E(\zeta(T)V(T))$. Hence it is valid to call $\zeta(t)$ the \emph{state price density process}.
\item (Exercises 4.9-4.11 [Shreve] Black-Scholes-Merton Equation)
For a European call with mature time $T$ and strike price $K$, the BSM price at time $t$ is 
\[
c(t,x)=xN(d_+)-Ke^{-r(T-t)}N(d_-),
\]
where
\[
d_{\pm} = \frac1{\sigma_1\sqrt{r}} \left( log\frac xK + (r\pm\frac12\sigma_1^2)r\right),
\]
However, the underlying asset is indeed a geometric Brownian motion with volatility 
\[\sigma_2>\sigma_1: dS(t)=\alpha S(t)dt + \sigma_2S(t)dW(t).\]
We set up a profolio with value denoted by $X(t)$. 
\\We remove cash from this portfolio at a rate $(\sigma_2^2-\sigma_1^2)S^2c_{xx}/2>0$. Hence, 
\[
dX = dc - c_xdS + r(X-c+Sc_x)dt - (\sigma_2^2-\sigma_1^2)S^2c_{xx}/2
\]
\begin{enumerate}
\item Show that $dX=rXdt$.
\item Write out the It$\hat o$'s formula for $d(e^{-rt}X(t))$. Deduce $dX=0$. \\This implies the existence of an arbitrage opportunity.
\end{enumerate}
\item (Exercise 4.20 [Shreve] Local Time)
The It$\hat o$'s Lemma in differential form says that \[df(x,t)=f_tdt+f_xdx+\frac12f_{xx}dxdx.\] Plug in $x=W(t)$ to get 
\begin{equation} 
df(W(t)) = f'(W(t))dW(t)+\frac12f''(W(t))dt.
\label{eq:dfW}
\end{equation}
\begin{enumerate}
\item Let $K>0$ a constant, and define $f(x)=max(x-K,0)$. Compute $f'(x), f''(x)$. Be careful about the points when either differential is not defined.
\item Show that Equation \ref{eq:dfW} does not hold for $f(x)=max(x-K,0)$. Hint: Consider taking expected values and integrals on both sides.
\end{enumerate}
To get some idea of what is going on here, we define a sequence of functions $\{f_n\}_{n=1}^{\infty}$ by
\[
f_n(x) = \begin{cases}
0 & x\leq K_{n-} \\
\frac n2(x-K)^2 + \frac12(x-K) +\frac1{8n} & K_{n-}\leq K\leq K_{n+}\\
x-K & x\geq K_{n+}
\end{cases}
\]
where $K_{n-}=K-1/({2n}), K_{n+}=K+1/({2n})$.
\begin{enumerate}
\addtocounter{enumii}{2}
\item Show that
\[
\lim_{n\to\infty} f_n(x) = max(x-K,0),
\]
and that 
\[
\lim_{n\to\infty} f'_n(x) = \begin{cases} 0 & x< K \\
1/2 & x=K \\
1 & x> K. \end{cases}
\]
\end{enumerate}
The value of $\lim_{n\to\infty}f'(x)$ at a single point will not matter when we integrate. We are constructing a continuous function $f_n(x)$ and $f_n'(x)$ is defined everywhere. Note further that $f_n''(x)$ is defined for $x\in\mathbb R\backslash\{K_+,K_-\}$, and $|f_n''(x)|$ is bounded above by $n$. Hence, the It$\hat o$'s Lemma applies to the function $f_n$ because the intergrals are well defined. 
\end{enumerate}
\end{document}